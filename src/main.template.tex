%%%%%%%%%%%%%%%%%
% This is an sample CV template created using altacv.cls
% (v1.3, 10 May 2020) written by LianTze Lim (liantze@gmail.com). Now compiles with pdfLaTeX, XeLaTeX and LuaLaTeX.
%
%% It may be distributed and/or modified under the
%% conditions of the LaTeX Project Public License, either version 1.3
%% of this license or (at your option) any later version.
%% The latest version of this license is in
%%    http://www.latex-project.org/lppl.txt
%% and version 1.3 or later is part of all distributions of LaTeX
%% version 2003/12/01 or later.
%%%%%%%%%%%%%%%%

%% If you need to pass whatever options to xcolor
\PassOptionsToPackage{dvipsnames}{xcolor}

%% If you are using \orcid or academicons
%% icons, make sure you have the academicons
%% option here, and compile with XeLaTeX
%% or LuaLaTeX.
% \documentclass[10pt,a4paper,academicons]{altacv}

%% Use the "normalphoto" option if you want a normal photo instead of cropped to a circle
% \documentclass[10pt,a4paper,normalphoto]{altacv}

\documentclass[10pt,a4paper,ragged2e,withhyper]{altacv}
%% AltaCV uses the fontawesome5 and academicons fonts
%% and packages.
%% See http://texdoc.net/pkg/fontawesome5 and http://texdoc.net/pkg/academicons for full list of symbols. You MUST compile with XeLaTeX or LuaLaTeX if you want to use academicons.

% Change the page layout if you need to
\geometry{left=$layout.margin.left$cm,right=$layout.margin.right$cm,top=$layout.margin.top$cm,bottom=$layout.margin.bottom$cm,columnsep=$layout.columnsep$cm}

% The paracol package lets you typeset columns of text in parallel
\usepackage{paracol}

% Change the font if you want to, depending on whether
% you're using pdflatex or xelatex/lualatex
\ifxetexorluatex
  % If using xelatex or lualatex:
  \setmainfont{Roboto Slab}
  \setsansfont{Lato}
  \renewcommand{\familydefault}{\sfdefault}
\else
  % If using pdflatex:
  \usepackage[rm]{roboto}
  \usepackage[defaultsans]{lato}
  % \usepackage{sourcesanspro}
  \renewcommand{\familydefault}{\sfdefault}
\fi

% Change the colours if you want to
[[ for colorname, colorhex in styling.colors.items() ]]
\definecolor{$colorname|camelize$}{HTML}{$colorhex$}
\colorlet{$colorname$}{$colorname|camelize$}
[[ endfor ]]

%\definecolor{SlateGrey}{HTML}{2E2E2E}
%\definecolor{LightGrey}{HTML}{666666}
%\definecolor{DarkPastelRed}{HTML}{450808}
%\definecolor{PastelRed}{HTML}{8F0D0D}
%\definecolor{GoldenEarth}{HTML}{E7D192}
%\colorlet{name}{black}
%\colorlet{tagline}{PastelRed}
%\colorlet{heading}{DarkPastelRed}
%\colorlet{headingrule}{GoldenEarth}
%\colorlet{subheading}{PastelRed}
%\colorlet{accent}{PastelRed}
%\colorlet{emphasis}{SlateGrey}
%\colorlet{body}{LightGrey}

% Change some fonts, if necessary
\renewcommand{\namefont}{\Huge\rmfamily\bfseries}
\renewcommand{\personalinfofont}{\footnotesize}
\renewcommand{\cvsectionfont}{\LARGE\rmfamily\bfseries}
\renewcommand{\cvsubsectionfont}{\large\bfseries}


% Change the bullets for itemize and rating marker
% for \cvskill if you want to
\renewcommand{\itemmarker}{{\small\textbullet}}
\renewcommand{\ratingmarker}{\faCircle}

%% sample.bib contains your publications
\addbibresource{sample.bib}

\begin{document}
\name{$name$}
\tagline{$tagline$}
%% You can add multiple photos on the left or right
[[ for photo in layout.photos]]
\photo[[ if photo.align == 'right' ]]R[[ else ]]L[[ endif ]]{$photo.width$cm}{$photo.filename$}
%\photoR{2.8cm}{$image$}
%\photoL{2.5cm}{$image$}
[[ endfor]]

\personalinfo{%
  % Not all of these are required!
  \email{$email$}
  \phone{$phone$}
  \mailaddress{$address$}
  [[ if location or country ]]
  \location{[[ if location ]]$location$[[ endif ]][[ if location and country ]], [[ endif ]][[ if country ]]$country$[[ endif ]]}
  [[ endif ]]
  [[ if homepage ]]
  \homepage{$homepage$}
  [[ endif ]]

  [[ if twitter ]]
  \twitter{$twitter$}
  [[ endif ]]
  [[ if linkeding ]]
  \linkedin{$linkedin$}
  [[ endif ]]
  [[ if github_username ]]
  \github{$github_username$}
  [[ endif ]]
  [[ if gitlab_username ]]
  \gitlab{$gitlab_username$}
  [[ endif ]]
  [[ if stackexchange_id ]]
  \printinfo{\faStackExchange}{StackExchange}[https://stackexchange.com/users/$stackexchange_id$?tab=accounts]
  [[ endif ]]
  [[ if telegram_username ]]
  \telegram{$telegram_username$}
  [[ endif ]]
  %% You MUST add the academicons option to \documentclass, then compile with LuaLaTeX or XeLaTeX, if you want to use \orcid or other academicons commands.
  % \orcid{0000-0000-0000-0000}
  %% You can add your own arbtrary detail with
  %% \printinfo{symbol}{detail}[optional hyperlink prefix]
  % \printinfo{\faPaw}{Hey ho!}[https://example.com/]
  %% Or you can declare your own field with
  %% \NewInfoFiled{fieldname}{symbol}[optional hyperlink prefix] and use it:
  % \NewInfoField{gitlab}{\faGitlab}[https://gitlab.com/]
  % \gitlab{your_id}
}

\makecvheader
%% Depending on your tastes, you may want to make fonts of itemize environments slightly smaller
% \AtBeginEnvironment{itemize}{\small}

%% Set the left/right column width ratio to 6:4.

\columnratio{$layout.columnratio$}

% Start a 2-column paracol. Both the left and right columns will automatically
% break across pages if things get too long.
\begin{paracol}{2}

[[ if experience_section.jobs.events|length > 0 ]]
\cvsection{$experience_section.jobs.title$}

[[ for job in experience_section.jobs.events ]]
\cvevent{$job.title$}{[[ if job.company ]]$job.company$[[ endif ]]}{[[if job.start_date ]]$job.start_date$[[ endif ]][[ if job.start_date and job.end_date]] -- [[ endif ]][[ if job.end_date ]]$job.end_date$[[ endif ]]}{[[ if job.location ]]$job.location$[[ endif ]]}
[[ if job.description ]]
$job.description$
[[ else ]]
\begin{itemize}
[[ for description in job.descriptions ]]
\item $description$
[[ endfor ]]
\end{itemize}
[[ endif ]]

[[ if loop.index < experience_section.jobs.events|length ]]
\divider
[[ endif ]]
[[ endfor ]]
[[ endif ]]

[[ if layout.new_page.left_column.after_experience ]]
\newpage
[[ else ]]
\medskip
[[ endif ]]

[[ if experience_section.projects.events|length > 0 ]]
\cvsection{$experience_section.projects.title$}

[[ for project in experience_section.projects.events ]]
\cvevent{$project.name$}{[[ if project.institution ]]$project.institution$[[ endif ]]}{[[ if project.start_date ]]$project.start_date$[[ endif ]][[ if project.start_date and project.end_date ]] -- [[ endif ]][[ if project.end_date ]]$project.end_date$[[ endif ]]}{[[ if project.location ]]$project.location$[[ endif ]]}
[[ if project.description ]]
$project.description$
[[ else ]]
\begin{itemize}
[[ for description in project.descriptions ]]
\item $description$
[[ endfor ]]
\end{itemize}
[[ endif ]]

[[ if loop.index < experience_section.projects.events|length ]]
\divider
[[ endif ]]
[[ endfor ]]
[[ endif ]]


[[ if layout.new_page.left_column.after_projects ]]
\newpage
[[ else ]]
\medskip
[[ endif ]]


\cvsection{$a_day_of_my_life.title$}

% Adapted from @Jake's answer from http://tex.stackexchange.com/a/82729/226
% \wheelchart{outer radius}{inner radius}{
% comma-separated list of value/text width/color/detail}
% hours/text-width/color-percent/text
\wheelchart{$layout.a_day_of_my_life_wheel.outer_radius$cm}{$layout.a_day_of_my_life_wheel.inner_radius$cm}{%
  [[ for activity in a_day_of_my_life.activities ]]
  $activity.hours$/$activity.text_width$em/ADayOfMyLifeWheel!$activity.percent_color$/{$activity.text$}[[ if loop.index < a_day_of_my_life.activities|length ]],[[ endif ]]
  [[ endfor ]]
}
%6/8em/accent!30/{Sleep,\\beautiful sleep},
%3/8em/accent!40/Hopeful novelist by night,
%8/8em/accent!60/Daytime job,
%2/10em/accent!100/Sports and relaxation,
%5/6em/accent!20/Spending time with family

[[ if layout.new_page.left_column.after_a_day_on_my_life ]]
\newpage
[[ else ]]
\medskip
[[ endif ]]


\cvsection{Publications}

\nocite{*}

\printbibliography[heading=pubtype,title={\printinfo{\faBook}{Books}},type=book]

\divider

\printbibliography[heading=pubtype,title={\printinfo{\faFile*[regular]}{Journal Articles}},type=article]

\divider

\printbibliography[heading=pubtype,title={\printinfo{\faUsers}{Conference Proceedings}},type=inproceedings]

%% Switch to the right column. This will now automatically move to the second
%% page if the content is too long.
\switchcolumn

\cvsection{My Life Philosophy}

\begin{quote}
``Something smart or heartfelt, preferably in one sentence.''
\end{quote}

\cvsection{Most Proud of}

\cvachievement{\faTrophy}{Fantastic Achievement}{and some details about it}

\divider

\cvachievement{\faHeartbeat}{Another achievement}{more details about it of course}

\divider

\cvachievement{\faHeartbeat}{Another achievement}{more details about it of course}

\cvsection{Strengths}

\cvtag{Hard-working}
\cvtag{Eye for detail}\\
\cvtag{Motivator \& Leader}

\divider\smallskip

\cvtag{C++}
\cvtag{Embedded Systems}\\
\cvtag{Statistical Analysis}

\cvsection{Languages}

\cvskill{English}{5}
\divider

\cvskill{Spanish}{4}
\divider

\cvskill{German}{3}

%% Yeah I didn't spend too much time making all the
%% spacing consistent... sorry. Use \smallskip, \medskip,
%% \bigskip, \vpsace etc to make ajustments.
\medskip

\cvsection{Education}

\cvevent{Ph.D.\ in Your Discipline}{Your University}{Sept 2002 -- June 2006}{}
Thesis title: Wonderful Research

\divider

\cvevent{M.Sc.\ in Your Discipline}{Your University}{Sept 2001 -- June 2002}{}

\divider

\cvevent{B.Sc.\ in Your Discipline}{Stanford University}{Sept 1998 -- June 2001}{}

% \divider

\cvsection{Referees}

% \cvref{name}{email}{mailing address}
\cvref{Prof.\ Alpha Beta}{Institute}{a.beta@university.edu}
{Address Line 1\\Address line 2}

\divider

\cvref{Prof.\ Gamma Delta}{Institute}{g.delta@university.edu}
{Address Line 1\\Address line 2}


\end{paracol}


\end{document}
