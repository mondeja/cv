%%%%%%%%%%%%%%%%%
% This is an sample CV template created using altacv.cls
% (v1.3, 10 May 2020) written by LianTze Lim (liantze@gmail.com). Now compiles with pdfLaTeX, XeLaTeX and LuaLaTeX.
%
%% It may be distributed and/or modified under the
%% conditions of the LaTeX Project Public License, either version 1.3
%% of this license or (at your option) any later version.
%% The latest version of this license is in
%%    http://www.latex-project.org/lppl.txt
%% and version 1.3 or later is part of all distributions of LaTeX
%% version 2003/12/01 or later.
%%%%%%%%%%%%%%%%

%% If you need to pass whatever options to xcolor
\PassOptionsToPackage{dvipsnames}{xcolor}

%% If you are using \orcid or academicons
%% icons, make sure you have the academicons
%% option here, and compile with XeLaTeX
%% or LuaLaTeX.
% \documentclass[10pt,a4paper,academicons]{altacv}

%% Use the "normalphoto" option if you want a normal photo instead of cropped to a circle
% \documentclass[10pt,a4paper,normalphoto]{altacv}

\documentclass[10pt,a4paper,ragged2e,withhyper\BLOCK{ if style.normalphoto },normalphoto\BLOCK{ endif }]{altacv}
%% AltaCV uses the fontawesome5 and academicons fonts
%% and packages.
%% See http://texdoc.net/pkg/fontawesome5 and http://texdoc.net/pkg/academicons for full list of symbols. You MUST compile with XeLaTeX or LuaLaTeX if you want to use academicons.

% Change the page layout if you need to
\geometry{left=\VAR{layout.margin.left}cm,right=\VAR{layout.margin.right}cm,top=\VAR{layout.margin.top}cm,bottom=\VAR{layout.margin.bottom}cm,columnsep=\VAR{layout.columnsep}cm}

% The paracol package lets you typeset columns of text in parallel
\usepackage{paracol}

% Change the font if you want to, depending on whether
% you're using pdflatex or xelatex/lualatex
\ifxetexorluatex
  % If using xelatex or lualatex:
  \setmainfont{Roboto Slab}
  \setsansfont{Lato}
  \renewcommand{\familydefault}{\sfdefault}
\else
  % If using pdflatex:
  \usepackage[rm]{roboto}
  \usepackage[defaultsans]{lato}
  % \usepackage{sourcesanspro}
  \renewcommand{\familydefault}{\sfdefault}
\fi

% Change the colours if you want to
\BLOCK{ for colorname, colorhex in style.colors.items() }
\definecolor{\VAR{colorname|camelize}}{HTML}{\VAR{colorhex}}
\colorlet{\VAR{colorname}}{\VAR{colorname|camelize}}
\BLOCK{ endfor }

%\definecolor{SlateGrey}{HTML}{2E2E2E}
%\definecolor{LightGrey}{HTML}{666666}
%\definecolor{DarkPastelRed}{HTML}{450808}
%\definecolor{PastelRed}{HTML}{8F0D0D}
%\definecolor{GoldenEarth}{HTML}{E7D192}
%\colorlet{name}{black}
%\colorlet{tagline}{PastelRed}
%\colorlet{heading}{DarkPastelRed}
%\colorlet{headingrule}{GoldenEarth}
%\colorlet{subheading}{PastelRed}
%\colorlet{accent}{PastelRed}
%\colorlet{emphasis}{SlateGrey}
%\colorlet{body}{LightGrey}

% Change some fonts, if necessary
\renewcommand{\namefont}{\Huge\rmfamily\bfseries}
\renewcommand{\personalinfofont}{\footnotesize}
\renewcommand{\cvsectionfont}{\LARGE\rmfamily\bfseries}
\renewcommand{\cvsubsectionfont}{\large\bfseries}


% Change the bullets for itemize and rating marker
% for \cvskill if you want to
\renewcommand{\itemmarker}{{\small\textbullet}}
\renewcommand{\ratingmarker}{\faCircle}

%% sample.bib contains your publications
\addbibresource{sample.bib}

\begin{document}
\name{\VAR{name}}
\tagline{\VAR{tagline}}
%% You can add multiple photos on the left or right
\BLOCK{ for photo in layout.photos }
\photo\BLOCK{ if photo.align == 'right' }R\BLOCK{ else }L\BLOCK{ endif }{\VAR{photo.width}cm}{\VAR{photo.filename}}
%\photoR{2.8cm}{$image$}
%\photoL{2.5cm}{$image$}
\BLOCK{ endfor }

\personalinfo{%
  % FontAwesome5 icons: http://www.ipgp.fr/~moguilny/LaTeX/fontawesome5Icons.pdf
  \BLOCK{ if email }
  \email{\VAR{email}}
  \BLOCK{ endif }
  \BLOCK{ if phone }
  \phone{\VAR{phone}}
  \BLOCK{ endif }
  \BLOCK{ if address }
  \mailaddress{\VAR{address}}
  \BLOCK{ endif }
  \BLOCK{ if location or country }
  \location{\BLOCK{ if location }\VAR{location}\BLOCK{ endif }\BLOCK{ if location and country }, \BLOCK{ endif }\BLOCK{ if country }\VAR{country}\BLOCK{ endif }}
  \BLOCK{ endif }
  \BLOCK{ if homepage }
  \homepage{\VAR{homepage}}
  \BLOCK{ endif }
  \BLOCK{ if twitter }
  \twitter{\VAR{twitter}}
  \BLOCK{ endif }
  \BLOCK{ if linkeding }
  \linkedin{\VAR{linkedin}}
  \BLOCK{ endif }
  \BLOCK{ if github_username }
  \github{\VAR{github_username}}
  \BLOCK{ endif }
  \BLOCK{ if gitlab_username }
  \gitlab{\VAR{gitlab_username}}
  \BLOCK{ endif }
  \BLOCK{ if stackexchange_id }
  \printinfo{\faStackExchange}{StackExchange}[https://stackexchange.com/users/\VAR{stackexchange_id}?tab=accounts]
  \BLOCK{ endif }
  \BLOCK{ if telegram_username }
  \telegram{\VAR{telegram_username}}
  \BLOCK{ endif }
  %% You MUST add the academicons option to \documentclass, then compile with LuaLaTeX or XeLaTeX, if you want to use \orcid or other academicons commands.
  %\orcid{0000-0000-0000-0000}
  %% You can add your own arbtrary detail with
  %% \printinfo{symbol}{detail}[optional hyperlink prefix]
  % \printinfo{\faPaw}{Hey ho!}[https://example.com/]
  %% Or you can declare your own field with
  %% \NewInfoFiled{fieldname}{symbol}[optional hyperlink prefix] and use it:
  % \NewInfoField{gitlab}{\faGitlab}[https://gitlab.com/]
  % \gitlab{your_id}
}

\makecvheader
%% Depending on your tastes, you may want to make fonts of itemize environments slightly smaller
% \AtBeginEnvironment{itemize}{\small}

%% Set the left/right column width ratio to 6:4.

\columnratio{\VAR{layout.columnratio}}

% Start a 2-column paracol. Both the left and right columns will automatically
% break across pages if things get too long.
\begin{paracol}{2}

\BLOCK{ if experience.jobs.events|length > 0 }
\cvsection{\VAR{experience.jobs.title}}

\BLOCK{ for job in experience.jobs.events }
\cvevent{\VAR{job.position}}{\BLOCK{ if job.company }\VAR{job.company}\BLOCK{ endif }}{\BLOCK{if job.start_date }\VAR{job.start_date}\BLOCK{ endif }\BLOCK{ if job.start_date and job.end_date} -- \BLOCK{ endif }\BLOCK{ if job.end_date }\VAR{job.end_date}\BLOCK{ endif }}{\BLOCK{ if job.location }\VAR{job.location}\BLOCK{ endif }}
\BLOCK{ if job.description }
\VAR{job.description}
\BLOCK{ else }
\begin{itemize}
\BLOCK{ for description in job.descriptions }
\item \VAR{description}
\BLOCK{ endfor }
\end{itemize}
\BLOCK{ endif }

\BLOCK{ if loop.index < experience.jobs.events|length }
\divider
\BLOCK{ endif }
\BLOCK{ endfor }
\BLOCK{ endif }

\BLOCK{ if layout.new_page.left_column.after_experience }
\newpage
\BLOCK{ else }
\medskip
\BLOCK{ endif }

\BLOCK{ if experience.projects.events|length > 0 }
\cvsection{\VAR{experience.projects.title}}

\BLOCK{ for project in experience.projects.events }
\cvevent{\VAR{project.name}}{\BLOCK{ if project.institution }\VAR{project.institution}\BLOCK{ endif }}{\BLOCK{ if project.start_date }\VAR{project.start_date}\BLOCK{ endif }\BLOCK{ if project.start_date and project.end_date } -- \BLOCK{ endif }\BLOCK{ if project.end_date }\VAR{project.end_date}\BLOCK{ endif }}{\BLOCK{ if project.location }\VAR{project.location}\BLOCK{ endif }}
\BLOCK{ if project.description }
\VAR{project.description}
\BLOCK{ else }
\begin{itemize}
\BLOCK{ for description in project.descriptions }
\item \VAR{description}
\BLOCK{ endfor }
\end{itemize}
\BLOCK{ endif }

\BLOCK{ if loop.index < experience.projects.events|length }
\divider
\BLOCK{ endif }
\BLOCK{ endfor }
\BLOCK{ endif }


\BLOCK{ if layout.new_page.left_column.after_projects }
\newpage
\BLOCK{ else }
\medskip
\BLOCK{ endif }


\cvsection{\VAR{a_day_of_my_life.title}}

% Adapted from @Jake's answer from http://tex.stackexchange.com/a/82729/226
% \wheelchart{outer radius}{inner radius}{
% comma-separated list of value/text width/color/detail}
% hours/text-width/color-percent/text
\wheelchart{\VAR{layout.a_day_of_my_life_wheel.outer_radius}cm}{\VAR{layout.a_day_of_my_life_wheel.inner_radius}cm}{%
  \BLOCK{ for activity in a_day_of_my_life.activities }
  \VAR{activity.hours}/\VAR{activity.text_width}em/ADayOfMyLifeWheel!\VAR{activity.percent_color}/{\VAR{activity.text}}\BLOCK{ if loop.index < a_day_of_my_life.activities|length },\BLOCK{ endif }
  \BLOCK{ endfor }
}
%6/8em/accent!30/{Sleep,\\beautiful sleep},
%3/8em/accent!40/Hopeful novelist by night,
%8/8em/accent!60/Daytime job,
%2/10em/accent!100/Sports and relaxation,
%5/6em/accent!20/Spending time with family

\BLOCK{ if layout.new_page.left_column.after_a_day_on_my_life }
\newpage
\BLOCK{ else }
\medskip
\BLOCK{ endif }


\cvsection{Publications}

\nocite{*}

\printbibliography[heading=pubtype,title={\printinfo{\faBook}{Books}},type=book]

\divider

\printbibliography[heading=pubtype,title={\printinfo{\faFile*[regular]}{Journal Articles}},type=article]

\divider

\printbibliography[heading=pubtype,title={\printinfo{\faUsers}{Conference Proceedings}},type=inproceedings]

%% Switch to the right column. This will now automatically move to the second
%% page if the content is too long.
\switchcolumn

\BLOCK{ if my_life_philosophy }
\cvsection{\VAR{my_life_philosophy.title}}
\begin{quote}
\VAR{my_life_philosophy.quote}
\end{quote}

\BLOCK{ if layout.new_page.right_column.after_my_life_philosophy }
\newpage
\BLOCK{ else }
\medskip
\BLOCK{ endif }
\BLOCK{ endif }

\BLOCK{ if most_proud_of }
\cvsection{\VAR{most_proud_of.title}}
\BLOCK{ for achievement in most_proud_of.achievements }
\cvachievement{\\VAR{achievement.icon}}{\VAR{achievement.title}}{\VAR{achievement.details}}

\BLOCK{ if loop.index < most_proud_of.achievements|length }
\divider
\BLOCK{ endif }

\BLOCK{ endfor }

\BLOCK{ if layout.new_page.right_column.after_most_proud_of }
\newpage
\BLOCK{ else }
\medskip
\BLOCK{ endif }
\BLOCK{ endif }


\BLOCK{ if psychology }
\cvsection{\VAR{psychology.title}}
\BLOCK{ for tag_group in psychology.tags_groups }

\BLOCK{ for tag in tag_group }
\BLOCK{ if tag == '\\n' }
\\
\BLOCK{ else }
\cvtag{\VAR{tag}}
\BLOCK{ endif }
\BLOCK{ endfor }

\BLOCK{ if loop.index < psychology.tags_groups|length }
\divider\smallskip
\BLOCK{ endif }
\BLOCK{ endfor }

\BLOCK{ if layout.new_page.right_column.after_psychology }
\newpage
\BLOCK{ else }
\medskip
\BLOCK{ endif }
\BLOCK{ endif }



\BLOCK{ if programming_languages }
\cvsection{\VAR{programming_languages.title}}

\BLOCK{ for language in programming_languages.languages }
\cvskill{\VAR{language.name}}{\VAR{language.stars}}
\BLOCK{ endfor }

\BLOCK{ if layout.new_page.right_column.after_programming_languages }
\newpage
\BLOCK{ else }
\medskip
\BLOCK{ endif }
\BLOCK{ endif }

\BLOCK{ if technologies }
\cvsection{\VAR{technologies.title}}
\BLOCK{ for tag_group in technologies.tags_groups }

\BLOCK{ for tag in tag_group }
\BLOCK{ if tag == '\\n' }
\\
\BLOCK{ else }
\cvtag{\VAR{tag}}
\BLOCK{ endif }
\BLOCK{ endfor }

\BLOCK{ if loop.index < technologies.tags_groups|length }
\divider\smallskip
\BLOCK{ endif }
\BLOCK{ endfor }

\BLOCK{ if layout.new_page.right_column.after_technologies }
\newpage
\BLOCK{ else }
\medskip
\BLOCK{ endif }
\BLOCK{ endif }


\BLOCK{ if languages }
\cvsection{\VAR{languages.title}}
\BLOCK{ for language in languages.languages }

\cvskill{\VAR{language.name}}{\VAR{language.stars}}

\BLOCK{ if loop.index < languages.languages|length }
\divider\smallskip
\BLOCK{ endif }

\BLOCK{ endfor }

\BLOCK{ if layout.new_page.right_column.after_languages }
\newpage
\BLOCK{ else }
\medskip
\BLOCK{ endif }
\BLOCK{ endif }


\BLOCK{ if education }
\cvsection{\VAR{ education.title }}

\BLOCK{ for event in education.events }

\cvevent{\VAR{event.degree}}{\VAR{event.institution} -- \VAR{event.location}}{\VAR{event.start_date} -- \VAR{event.end_date}}{}

\BLOCK{ if loop.index < education.events|length }
\divider
\BLOCK{ endif }

\BLOCK{ endfor }
\divider

\BLOCK{ endif }

% \divider

\cvsection{Referees}

% \cvref{name}{email}{mailing address}
\cvref{Prof.\ Alpha Beta}{Institute}{a.beta@university.edu}
{Address Line 1\\Address line 2}

\divider

\cvref{Prof.\ Gamma Delta}{Institute}{g.delta@university.edu}
{Address Line 1\\Address line 2}


\end{paracol}


\end{document}
